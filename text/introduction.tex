
\chapter{Introduction}
Pattern matching is widely needed because of its usage in the most of the modern databases. Whenever user needs to find specific data matching, his requirements it can be seen as if the query was a pattern that needs to be matched in the data. Imagine an example query: astronomer wants to find all stars in specific part of the night sky with luminosity higher than \textit{x}, or meteorologist needs to check coordinates of all possible tornadoes on Earth where a tornado can be specified by different humidity, wind change, temperature. \cite{detectRivers}

This search can be divided into two categories depending on the presence of errors. If no mismatches are allowed it is called \textit{exact pattern matching} when otherwise the problem is called \textit{approximate pattern matching}. Approximate version of pattern matching can have numerous advantages like allowing search with the spelling errors or only partially specified search. 

Majority of the research in this field of both exact and approximate pattern matching was done primarily on one and two dimensional arrays where algorithms that can process higher dimensional arrays are usually low dimensional algorithms converted into high dimensional array spaces or use methods that can reduce dimensionality of both data and patterns.

Modern databases (e.g. SciDB) contain ever increasing number of data because they store whole range of informations for example satellites imagery \cite{scidbarch}. These data are stored in sparse arrays and thus need to be either preprocessed into special structure like Coordinate format, Compressed Sparse Row format \cite{saad1990sparskit}, which are commonly used for computations in linear algebra, or a binary mask can be used, which is used for achieving of efficient data storage, and is the implemented solution in this work.

The main goal of this work is to analyse current solutions of pattern matching while implementing three of these proposed in the paper \cite{mdApproxPM}. Next step will focus on creating of a new algorithm based on these and the usage of machine learning methods. Another condition to fulfil is for the algorithm to be able to process data in SciDB binary format. 

Next chapter called Related Work is dedicated to presenting similar papers, thesis and work done in the field of pattern matching. The section after is focused on analysis and explaining of all important terms used in this work. Third section deals with the implementation details and mentions how to use (compile and run) created programs. Second to last chapter specifies and explains the results while the last chapter briefly summarizes what was accomplished and defines the future work.

\setsecnumdepth{all}
\chapter{Related work}
Theme of this thesis is inspired by the work of Baeza-Yates and Navarro in the field of pattern matching algorithms. In their papers \cite{mdApproxPM}, \cite{fast2DapproxPM}, \cite{fastMDApproxPM}, they extend different types of approximate pattern matching, such as in the way of reducing dimensionality by creating filters, or by creating new similarity measures for strings. When mentioning filters, authors of the first string filter Bird and Baker, and their successors J. K{\" a}rkk{\" a}inen and Ukkonen can not be forgotten.

Between most important papers and work done in this field belong R. Baeza-Yates with his work on similarity of two-dimensional strings written in 1998 \cite{sim2Dstrings} which is focused on computing edit distance between two images while using new similarity measures. Another fundamental paper by R. Baeza-Yates and G. Navarro \cite{fast2DapproxPM} is concerned with fast two-dimensional approximate pattern matching (also from 1998) where authors construct search algorithm for approximate pattern matching based on filtering of one dimensional multi patterns.

Baeza-Yates also worked with C. Perleberg on the paper published in 1992 \cite{fastApproxStringMatching} dealing with the question of fast and practical approximate pattern matching which they solved by matching strings with their mismatches based on arithmetical operations. Errors here are based on partitioning the pattern. In the next paper written in 1999 by Baeza-Yates and Navarro named Fast multi-dimensional approximate string matching \cite{fastMDApproxPM} they are extending two dimensional algorithm into \textit{n} dimensions by creating sub-linear time searching algorithm. This solution turned out to be better than using the dynamic programming.

Lot of progress was made in this field by J. K{\" a}rkk{\" a}inen and E. Ukkonen with their concentration on filters to quickly discard a large amount of data \cite{karkoptimal}. In their work from 1994 they concentrate on two and higher dimensional pattern matching in optimal expected time where they try placing a static grid of test points into the text and then eliminate as many potential occurrences of incorrect patterns as possible.

Among other scientists dealing with the question of efficient two dimensional pattern matching belong K. Krithivasan and R. Sitalakshmi \cite{effPMerr} who were focused mainly on solving two dimensional pattern matching in the presence of errors as published in their paper from 1987.

Also, there must be mentioned work done in the field of sequence alignment and general work with DNA sequences which is very similar to the task of pattern matching. Of all the scientists involved there can be named P. Sellers with his paper written in 1980 and named: The theory and computation of evolutionary distances: pattern recognition \cite{evolDistances}, where he is focused on finding pattern similarities between two sequences with the computation time being a product of the sequences length.

Lastly in the book Jewels of Stringology: Text Algorithms published in 2003 \cite{stringJewels} with authors M Crochemore, W Rytter there are presented various basic algorithms that solves the question of pattern matching but only when considering strings.

All of the papers and work mentioned above are focused mainly on string pattern matching and maximally two dimensional spaces, where when creating a solution for higher dimensional space it is usually generalized version for one or two dimensional space.

