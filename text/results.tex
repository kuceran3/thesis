\chapter{Results}

In the table \ref{tab_chunk} there can be seen how using the binary format affects the size of the used data files.
%tabulka se zmensenim

\begin{table}
\centering
\begin{tabular} {| c | c | c | c |}

\hline
Density (\%) & 100 & 50 & 2 \\
\hline
Size Before (MB) & 239.6 & 235.7 & 284.3 \\
\hline
Size After (MB) & 39.8 & 38.1 & 109.8 \\
\hline
\hline
New Size (\%) & 16.61 & 16.16 & 38.62  \\
\hline
\end{tabular}
\caption{Size of input files before and after chunking.}
\label{tab_chunk}
\end{table}



File 256MB with 50\% density, 7 results.  Time (sec).\\
\begin{tabular} {| c | c | c | c | c | c | c | c|}

\hline
  & \multicolumn{7}{|c|}{Algorithm} \\
\hline
Errors & BE & SE & BA & FFS & SF & HSF & HSF1 \\
\hline
0 & 42.195 & 22.0131 & 267.855 & 18.8743 & 18.9291 & 28.2716 & 24.8698 \\
4 & - & - & 265.971 & 154.847 & 153.403 & 269.145 & 213.794 \\
16 & - & - & 668.237 & 113.360 & 110.024 & 236.046 & 224.018 \\
64 & - & - & 1857.26 & 98.4649 & 98.8414 & 333.046 & 480.422 \\
\hline
\end{tabular}
\\

Errors - number of errors allowed
BE - brute exact
SE - skip exact
BA - brute approximate
FFS - Fast filter split
SF - stricter filter
HSF - Hash SF
HSF1 - Hash once SF
HSFR - Hash SF rolling
